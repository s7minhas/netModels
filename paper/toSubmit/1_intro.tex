Network analysis provides a way to represent and study ``relational data'', that is data with characteristics extending beyond those of the individual. Data structures that define relations between pairs of actors are ubiquitous in political science -- examples include the study of events such as legislation cosponsorship, trade, interstate conflict, and the formation of international agreements. The dominant paradigm for dealing with such data, however, is not a network approach but rather a dyadic design, in which an interaction between a pair of actors is considered independent of interactions between any other pair in the system. To highlight the ubiquity of this approach the following represent just a sampling of the articles published from the 1980s to the present in the American Journal of Political Science (AJPS) and American Political Science Review (APSR) that assume dyadic independence: \citet{dixon:1983,mansfield:etal:2000,lemke:reed:2001a,mitchell:2002,dafoe:2011a,fuhrmann:sechser:2014,carnegie:2014}.

The implication of this assumption is that when, for example, Vietnam and the United States decide to form a trade agreement, they make this decision independently of what they have done with other countries and what other countries in the international system have done among themselves.\footnote{There has been significant work done on treaty formation that would challenge this claim, e.g., see \citet{manger:etal:2012,kinne:2013}.} An even stronger assumption is that Japan declaring war against the United States is independent of the decision of the United States to go to war against Japan.\footnote{\citet{maoz:etal:2006,minhas:etal:2016} would each note the importance of taking into account network dynamics in the study of interstate conflict.} A common refrain from those that favor the dyadic approach is that many events are only bilateral (\citealt{diehl:wright:2016}), thus alleviating the need for an approach that incorporates interdependencies between observations. However, even bilateral events and processes take place within a broader system, and occurrences in one part of the system may be dependent upon events in another. At a minimum, we don't know whether independence of events and processes characterizes what we observe. 

In this article, we introduce the additive and multiplicative effects (AME) model and compare it to two popular alternatives: the latent space model (LSM) and exponential random graph model (ERGM). The AME approach to network modeling is a flexible framework that can be used to estimate many different types of cross-sectional and longitudinal networks with binary, ordinal, or continuous edges within a generalized linear model framework. Our approach addresses ways in which observations can be interdependent while still allowing scholars to focus on examining theories that may only be relevant in the monadic or dyadic level. Further, at the network level it accounts for nodal and dyadic dependence patterns, and provides a descriptive visualization of higher-order dependencies such as homophily and stochastic equivalence. 

The paper is organized as follows, we begin by briefly discussing the difficulties in studying dyadic data through approaches that assume observational independence; and how ERGMs provide one avenue to to address these issues. Then we introduce the AME framework in two steps. We first discuss nodal and dyadic dependencies that may lead to non-iid observations and show how the additive effects portion of AME can be used to model these dependencies. Similarly, in the second step, we discuss how the multiplicative effects portion of the AME framework can be used to effectively model third order effects, particularly when compared to alternatives such as the LSM. We conclude with an application of these models on a cross-sectional network measuring collaborations during the policy design of the Swiss CO$_{2}$ act. We show that AME provides a superior goodness of fit to the data in terms of ability to predict linkages and capture network dependencies. 
\\