\section*{\textbf{Conclusion}}

The AME approach to estimation and inference in network data provides a number of benefits over extant alternatives in political science. Specifically, it provides a modeling framework for dyadic data that is based on familiar statistical tools such as linear regression, GLM, random effects, and factor models.\footnote{A number of related approaches have been developed that also stem from latent variable models: \citet{sewell:chen:2015,gollini:murphy:2016,durante:etal:2017,kao:etal:2018}. Each of these approaches differ in how they construct the latent variable term to account for third-order dependencies, but they each are based off of a similar framework as the model we present here. We hope that this paper motivates further interest in exploring the utility of latent variable models to studying networks in political science.} Further we have shown that alternatives such as the LSM complicate parameter interpretation due to the construction of the latent variable term. The benefit of AME is that its focus intersects with the interest of most IR scholars, which is primarily on the effects of exogenous covariates. For researchers in the social sciences this is of primary interest, as many studies that employ relational data still have conceptualizations that are monadic or dyadic in nature.

ERGMs are best suited for cases in which scholars are interested in studying the role that particular types of node- and dyad-based network configurations play in generating the network. Though valuable this is often orthogonal to the interest of most researchers who are focused on studying the affect of a particular exogenous variable, such as democracy, on a dyadic variable like conflict while simply accounting for network dependencies. Additionally, through the application dataset utilized herein we show that the AME approach outperforms both ERGM and LSM in out-of-sample prediction, and also is better able to capture network dependencies than the LSM.

More broadly, relational data structures are composed of actors that are part of a system.\footnote{Additionally, in most political science applications, we are interested in how actors behave towards each other over time. Accounting for repeated interaction within AME can be done by including time-dependent regression terms such as lags of the dependent variable or simply time-varying regression parameters.} It is unlikely that this system can be viewed simply as a collection of isolated actors or pairs of actors. The assumption  that dependencies between observations occur can at the very least be examined. Failure to take into account interdependencies leads to biased parameter estimates and poor fitting models. By using standard diagnostics such as shown in Figure~\ref{fig:ergmAmePerf}, one can easily assess whether an assumption of independence is reasonable. We stress this point because a common misunderstanding that seems to have emerged within the social science literature relying on dyadic data is that a network based approach is only necessary if one has theoretical explanations that extend beyond the dyadic. This is not at all the case and findings that continue to employ a dyadic design may misrepresent the effects of the very variables that they are interested in. The AME approach that we have detailed here provides a statistically familiar way for scholars to account for unobserved network structures in relational data. 