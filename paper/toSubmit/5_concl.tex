\section*{\textbf{Conclusion}}

The AME approach to estimation and inference in network data provides a number of benefits over alternative approaches. Specifically, it provides a modeling framework for dyadic data that is based on familiar statistical tools such as linear regression, GLM, random effects, and factor models. We have an understanding of how each of these tools work, they are numerically more stable than ERGM approaches, and more general than alternative latent variable models. Further the estimation procedure utilized in AME avoids complicating interpretation of parameter estimates for exogenous variables. For researchers in the social sciences this is of primary interest, as many studies that employ relational data still have conceptualizations that are monadic or dyadic in nature. Additionally, through the application dataset utilized herein we show that the AME approach outperforms both ERGM and LSM in out-of-sample prediction, and also is better able to capture network dependencies than the LSM.

More broadly, relational data structures are composed of actors that are part of a system. It is unlikely that this system can be viewed simply as a collection of isolated actors or pairs of actors. The assumption  that dependencies between observations occur can at the very least be examined. Failure to take into account interdependencies leads to biased parameter estimates and poor fitting models. By using standard diagnostics such as shown in Figure~\ref{fig:ergmAmePerf}, one can easily assess whether an assumption of independence is reasonable. We stress this point because a common misunderstanding that seems to have emerged within the social science literature relying on dyadic data is that a network based approach is only necessary if one has theoretical explanations that extend beyond the dyadic. This is not at all the case and findings that continue to employ a dyadic design may misrepresent the effects of the very variables that they are interested in. The AME approach that we have detailed here provides a statistically familiar way for scholars to account for unobserved network structures in relational data. Additionally, through this approach we can visualize these dependencies in order to better learn about the network patterns that remain in the event of interest after having accounted for observed covariates.

When compared to other network based approaches, AME is easier to specify and utilize. 