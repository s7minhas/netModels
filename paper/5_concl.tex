\section{\textbf{Conclusion}}

The AME approach to estimation and inference in network data provides a number of benefits over alternative approaches. Specifically, it provides a modeling framework for dyadic data that is based on familiar statistical tools such as linear regression, GLM, random effects, and factor models. We have an understanding of how each of these tools work, they are numerically more stable than ERGM approaches, and more general than alternative latent variable models such as the latent distance or class frameworks. Further the estimation procedure utilized in AME avoids confounding the effects of nodal and dyadic covariates with actor positions in the latent space as the latent distance variable does. For researchers in international relations and more broadly across political science this is of primary interest, as many studies that employ relational data still have conceptualizations  that are monadic or dyadic in nature. Additionally, through the application dataset utilized herein we show that the AME approach outperforms both ERGM and latent distance models in out-of-sample prediction, and also is better able to capture network dependencies than the latent distance model. 

More broadly, relational data structures are composed of actors that are part of a system. It is unlikely that this system can be viewed simply as a collection of isolated actors or pairs of actors. The assumption  that dependencies between observations occur can at the very least be examined. Failure to take into account interdependencies leads to biased parameter estimates and poor fitting models. By using standard diagnostics such as shown in  Figures~\ref{fig:gofAll} and \ref{fig:ergmAmePerf}, one can easily assess whether an assumption of independence is reasonable. We stress this point because a common misunderstanding that seems to have emerged within the political science literature relying on dyadic data is that a network based approach is only necessary if one has theoretical explanations that extend beyond the dyadic. This is not at all the case and findings that continue to employ a dyadic design may misrepresent the effects of the very variables that they are interested in.  Calls to jettison Latent Space approaches are misguided, ignoring the benefits of an examination of network structures that may not be completely observed. At the same time, the latent space model permits a  flexible approach to specifying the higher-order dependencies that represent how networks may structure behavior at the dyadic and individual levels.  

When compared to other network based approaches such as ERGM, AME is easier to specify and utilize. It is also more straightforward to interpret since it does not require interpretation of unusual features such as \textit{three-stars} which fall outside of the normal language for discussing social science. Further, the \pkg{amen} package  also facilitates the modeling of longitudinal network data. In sum, excuses for continuing to treat relational data as conditionally independent are no longer valid. At the same time it is too early to focus only on ERGM-based approaches.