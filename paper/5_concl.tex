\section{\textbf{Conclusion}}

The AME approach that we introduce here provides a number of benefits over alternative latent space approaches. It provides for the representation of a variety of higher order dependence structures, and the estimation procedure utilized in AME avoids confounding the effects of nodal and dyadic covariates with actor positions in the latent space. For researchers in IR and more broadly across political science this is of primary interest, as we suspect that many of the theories they actually have of relational data are monadic or dyadic in nature. Additionally, in the application section, we show that the AME approach far outperforms the LSM model as implemented in the \pkg{latentnet} package both in terms of predictive performance and the capturing of network dependencies. 

More broadly, as we have noted throughout this paper relational data structures are composed of actors that are part of a system, it is highly unlikely that this system can be viewed simply as a collection of isolated dyads. The assumption should be that dependencies between observations likely exist and at the very least we should test for them. Failure to take into account interdependencies leads to biased parameter estimates and a model that will likely be a very poor fit to the data. By using diagnostics such as the ones we discussed in Figures~\ref{fig:gofAll} and \ref{fig:ergmAmePerf}, one can easily assess whether an assumption of independence is reasonable and from there decide on the simplest approach to proceed. We stress this point repeatedly because a common misunderstanding that seems to have emerged within the political science literature relying on dyadic data is that a network based approach is only necessary if one has theoretical explanations that extend beyond the dyadic. This is not at all the case and findings that continue to employ a dyadic design may misrepresent the effects of the very variables that they are interested in. 

Additionally, when compared to other network based approaches such as ERGM, AME is vastly easier to actually specify and utilize. Further the \pkg{amen} package facilitates the modeling of longitudinal network data. In sum, excuses for continuing to treat relational data as conditionally independent are becoming unnecessary. 