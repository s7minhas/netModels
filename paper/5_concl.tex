\section{\textbf{Conclusion}}

The AME approach that we introduce here is a flexible scheme that can be used to handle relational data structures arising from a variety of distributions and is also able to estimate models on longitudinal relational data structures. As we have noted throughout this paper relational data structures are composed of actors that are part of a system, it is highly unlikely that this system can be viewed simply as a collection of isolated dyads. The assumption should be that dependencies between observations likely exist and at the very least we should test for them. Failure to take into account interdependencies leads to biased parameter estimates and a model that will likely be a very poor fit to the data. Further through using diagnostics such as the ones we discussed in Figures~\ref{fig:gofAll} and \ref{fig:ergmAmePerf}, one can easily assess whether an assumption of independence is reasonable and from there decide on the simplest approach to proceed.

