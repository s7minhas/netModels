\section{\textbf{Conclusion}}

The AME approach to estimation and inference in network data provides a number of benefits over alternative approaches. It provides for the representation of a variety of higher order dependence structures. These can be empirically estimated and inference on them undertaken.
The estimation procedure utilized in AME avoids confounding the effects of nodal and dyadic covariates with actor positions in the latent space. For researchers in IR and more broadly across political science this is of primary interest, because many studies employ relational data are monadic or dyadic in nature. We show that the AME approach far outperforms the LSM model as implemented in the \pkg{latentnet} package both in terms of predictive performance and the capturing of network dependencies. It also has advantages over the ERGM approaches in terms of computation and scope. 

More broadly,  relational data structures are composed of actors that are part of a system. It is unlikely that this system can be viewed simply as a collection of isolated actors or pairs of actors. The assumption  that dependencies between observations occur can at the very least be examined. Failure to take into account interdependencies leads to biased parameter estimates and 
poor fitting models. By using standard diagnostics such as shown in  Figures~\ref{fig:gofAll} and \ref{fig:ergmAmePerf}, one can easily assess whether an assumption of independence is reasonable. We stress this point   because a common misunderstanding that seems to have emerged within the political science literature relying on dyadic data is that a network based approach is only necessary if one has theoretical explanations that extend beyond the dyadic. This is not at all the case and findings that continue to employ a dyadic design may misrepresent the effects of the very variables that they are interested in. 

When compared to other network based approaches such as ERGM, AME is   easier to   specify and utilize. And it is more straightforward to interpret since it does not require interpretation of unusual features such as \textit{three-stars} which fall outside of the normal language for discussing social science.  Further the \pkg{amen} package facilitates the modeling of longitudinal network data. In sum, excuses for continuing to treat relational data as conditionally independent are becoming unnecessary. 