Network analysis provides a way to represent and study ``relational data'', that is data, with characteristics extending beyond those of the individual, or in the parlance of International Relations (IR), characteristics beyond the monadic. Data structures that extend beyond the monadic level are quite simply the norm when it comes to the study of events such as trade, interstate conflict, or the formation of international agreements. The dominant paradigm in IR for dealing with data structures of this sort, however, is not a network approach but rather a dyadic design, in which an interaction between a pair of countries is considered independent of interactions between any other pair in the system.\footnote{To highlight the ubiquity of this approach the following represent just a sampling of the articles published from the 1980s to the present in the American Journal of Political Science (AJPS) and American Political Science Review (APSR) that assume dyadic independence: \citet{dixon:1983,mansfield:etal:2000,lemke:reed:2001a,mitchell:2002,dafoe:2011a,fuhrmann:sechser:2014,carnegie:2014}.} 

The implication of this assumption is that when, for example, Vietnam and the United States decide to form a trade agreement they make this decision independently of what they have done with other countries and what other countries in the international system have done amongst themselves.\footnote{There has been plenty of work done on treaty formation that would challenge this claim, e.g., see \citet{manger:etal:2012,kinne:2013}.} An even harder assumption to maintain is that Japan declaring war against the United States is independent of the decision of the United States going to war against Japan.\footnote{\citet{maoz:etal:2006,ward:etal:2007,minhas:etal:2016} would each note the importance of taking into account network dynamics in the study of interstate conflict.} A common refrain from those that continue to favor the dyadic approach is that many events are not multilateral \citep{diehl:wright:2016}, thus alleviating the need for an approach that incorporates interdependencies between observations. The network perspective, however, is that even the bilateral events we study are taking place within a broader system, and what takes place in one part of that system may be dependent upon another. 

% Shouldn't ignore iid assumption: 

% $\{i, j, k, l \}$

% $L(\theta | y, X) = f(y_{ij}, y_{ik}, y_{il}, \ldots, y_{li}, y_{lj}, y_{lk} | x_{ij}, \theta)$

% $L(\theta | y, X) = \prod f(y | X, \theta)$

The potential for interdependence between observations poses a challenge to statistical modeling as the assumption made by standard approaches used across the social sciences is that observations are, at least, conditionally independent \citep{snijders:2011}. The consequence of ignoring this assumption have been frequently noted within the political science literature already.\footnote{For example, see \citet{beck:etal:1998,signorino:1999,hoff:ward:2004,franzese:hayes:2007,cranmer:desmarais:2011,erikson:pinto:2014}.} More relevant is the fact that a wealth of research from other disciplines would argue that carrying the independence assumption into a study with relational data is misguided and likely to lead to biased inferences.\footnote{From Computer Science see: \citet{bonabeau:2002,brandes:erlebach:2005}. From Economics see: \citet{goyal:2012,jackson:2014}. From Psychology see: \citet{pattison:wasserman:1999,kenny:etal:2006}. From Statistics see: \citet{snijders:1996,hoff:etal:2002}.} 

Despite the hesitation among some in the discipline to adopt network analytic approaches, in recent years we have at least seen a greater level of interest in understanding these approaches. For instance, in the past year special issues focused on the application of a variety of network approaches have come out in the Journal of Peace Research and International Studies Quarterly. Particularly notable is a piece by \citet{cranmer:etal:2016} that provides an overview and comparison of a handful of widely used network based approaches, specifically, they focus on the exponential random graph model (ERGM), the multiple regression quadratic assignment procedure (MRQAP), and a latent distance approach developed by \citet{hoff:etal:2002}. Their discussion around the differences in these approaches and their empirical comparison of them is extremely valuable and necessary, at the same time, they overlook a decade worth of developments that latent variable models have undergone. This is particularly relevant in the context of providing an overview for the field as by focusing on the results from an earlier attempt at a latent variable model, they end up overlooking much of the work that has actually been done using this type of approach in political science. The principal latent variable approach used in political science is the general bilinear mixed-effects (GBME) model, developed by \citet{hoff:2005}. Examples of political science applications of the GBME include \citet{hoff:ward:2004,ward:etal:2007,metternich:etal:2015}, we are not aware of any political science applications using the earlier latent distance approach.\footnote{The software for the latent factor model used in these papers has been available since 2004 at the following address: \url{http://www.stat.washington.edu/people/pdhoff/Code/hoff_2005_jasa/}.} As \citet{hoff:2008} notes, the distinction between the latent distance and factor models is consequential when accounting for higher order interdependencies.   

In this paper we introduce a more general form of the latent factor model and show that this approach provides a far superior goodness of fit to the application presented in \citet{cranmer:etal:2016} than any of the models they discuss. This approach has been operationalized in the Additive and Multiplicative Effects Models for Networks and Relational Data, \pkg{amen}, package available on \href{https://cran.r-project.org/web/packages/amen/index.html}{CRAN}.\footnote{This package was published on CRAN in mid-2015.} The \pkg{amen} package provides for the estimation of many different relational data structures (e.g., binomial, gaussian, and ordinal edges), and it can estimate models for both cross-sectional and longitudinal networks. The rest of this paper proceeds as follows, we introduce the modelling framework used in the \pkg{amen} package, compare it to previous implementations of latent variable approaches, and then end by showing how this approach fits the application presented in \citet{cranmer:etal:2016}. We believe that this modelling framework can provide a flexible and easy to use scheme through which scholars can study relational data. It addresses the issue of interdependence while still allowing scholars to test theories that may only be relevant in the monadic or dyadic level.\footnote{We are aware of the emerging critique from \citet{jones:etal:2016} that latent variable models do not reduce the possibility for inferential error. However, like \citet{cranmer:etal:2016} they use an earlier version of the latent variable approach that, to our knowledge, no one in Political Science has actually applied. Further in our replication of the application presented in Cranmer et al. we show that the approach we present here produces a far better fit of the data than alternative network approaches, while also returning parameter estimates in line with the theoretical arguments described in the original paper.} At the network level it provides estimates of degree related effects, reciprocity, and provides a descriptive visualization of higher order dependencies such as transitivity. We agree with \citet{cranmer:etal:2016} that it does not allow for the explicit testing of parameter such as the number of two-paths in a network system, however, we would note that for the most part those parameters are of little use in the types of theories that political scientists tend to develop. 

% One approach has been to incorporate endogenous network parameters into extant modelling approaches (e.g., \citealp{fowler:2006, maoz:2009a}). Alternatively, we can employ a network based approach that accounts for and provides a way to measure interdependence. Latent space approach, exponential random graph model. 

% We eschew a long discussion of the technical details behind each of these approaches because \citet{cranmer:etal:2016} have already provided it. Our primary motivation here is to better introduce a key modeling framework that has been use in Political Science that they did not address. 