Network analysis provides a way to represent and study ``relational data'', that is data, with characteristics extending beyond those of the individual, or in the parlance of International Relations (IR), characteristics beyond the monadic. Data structures that extend beyond the monadic level are quite simply the norm when it comes to the study of events such as trade, interstate conflict, or the formation of international agreements. The dominant paradigm in  international relations for dealing with such data structures , however, is not a network approach but rather a dyadic design, in which an interaction between a pair of countries is considered independent of interactions between any other pair in the system.\footnote{To highlight the ubiquity of this approach the following represent just a sampling of the articles published from the 1980s to the present in the American Journal of Political Science (AJPS) and American Political Science Review (APSR) that assume dyadic independence: \citet{dixon:1983,mansfield:etal:2000,lemke:reed:2001a,mitchell:2002,dafoe:2011a,fuhrmann:sechser:2014,carnegie:2014}. See also XXXXX, iSQ special issue on networks.} 

The implication of this assumption is that when, for example, Vietnam and the United States decide to form a trade agreement they make this decision independently of what they have done with other countries and what other countries in the international system have done among themselves.\footnote{There has been plenty of work done on treaty formation that would challenge this claim, e.g., see \citet{manger:etal:2012,kinne:2013}.} An even stronger assumption is that Japan declaring war against the United States is independent of the decision of the United States going to war against Japan.\footnote{\citet{maoz:etal:2006,ward:etal:2007,minhas:etal:2016} would each note the importance of taking into account network dynamics in the study of interstate conflict.} A common refrain from those that favor the dyadic approach is that many events are not multilateral \citep{diehl:wright:2016}, thus alleviating the need for an approach that incorporates interdependencies between observations. This is clearly wrong. The network perspective asserts that even the bilateral events and processes take place within a broader system. What takes place in one part of the system may be dependent upon another. At a minimum we don't know whether independence of events and processes characterizes what we observe. At worst, we should examine this assertion.  

% Yet this perspective assumes that the international system is a collection of isolated dyads rather than a broader interconnected system. This means that the relationship for any given pair of states is considered independent of what's occurring anywhere else in the world. 

The potential for interdependence between observations poses a challenge to theoretical as well as statistical modeling since the assumption made by standard approaches used across the social sciences is that observations are, at least, conditionally independent \citep{snijders:2011}. The consequence of ignoring this assumption have been frequently noted within the political science literature already. For example, see \citet{beck:etal:1998,signorino:1999,hoff:ward:2004,franzese:hayes:2007,cranmer:desmarais:2011,erikson:pinto:2014}.  More relevant is the fact that a wealth of research from other disciplines suggests that carrying the independence assumption into a study with relational data is misguided and most often leads to biased inferences.\footnote{From Computer Science see: \citet{bonabeau:2002,brandes:erlebach:2005}. From Economics see: \citet{goyal:2012,jackson:2014}. From Psychology see: \citet{pattison:wasserman:1999,kenny:etal:2006}. From Statistics and Sociology see: \citet{snijders:1996,hoff:etal:2002}.} 


Despite the hesitation among some in the discipline to adopt network analytic approaches, in recent years there has been a greater level of interest in understanding these approaches. For instance, in the past year special issues focused on the application of a variety of network approaches have come out in the \textit{Journal of Peace Research} and \textit{International Studies Quarterly}. Particularly notable is a recent overview and comparison of a handful of network based inferential models  \citet{cranmer:etal:2016}.

Specifically, they focus on the exponential random graph model (ERGM), the multiple regression quadratic assignment procedure (MRQAP), and a latent distance approach developed by \citet{hoff:etal:2002}. Their discussion around the differences among these approaches and their empirical comparison of them is  valuable. At the same time, they overlook a decade worth of developments that latent variable model approaches have undergone. This is especially relevant in the context of providing an overview for the field by focusing on the results from one early attempt at a latent variable model, they end up overlooking a substantial amount of work using this type of approach in political science.\footnote{Indeed, in so far as we can tell, no one in political science has actually employed the Euclidean approach they summarize.} The principal latent variable approach used in political science has been the general bilinear mixed-effects (GBME) model developed by \citet{hoff:2005}. Examples of political science applications of the GBME include \citet{hoff:ward:2004,ward:etal:2007,cao:2009,cao:2010, cao:2012,breunig:etal:2012,ward:etal:2012,cao:ward:2014,metternich:etal:2015,greenhill:2015}, we are not aware of any political science applications using the latent distance approach.\footnote{The code necessary to run the GBME has been available since 2004 at the following address: \url{http://www.stat.washington.edu/people/pdhoff/Code/hoff_2005_jasa/}.} As \citet{hoff:2008} shows both empirically and mathematically, the distinction between the latent distance and factor models, the next step of the GBME, is consequential when accounting for higher-order interdependencies, a point overlooked by Cranmer et al. (2016).

In this article, we introduce the additive and multiplicative effects network model (AMEN). To highlight the benefits of this approach we estimate this model using data from  the application presented in \citet{cranmer:etal:2016} and compare it to the other models presented in that article. By doing so we are able to show that AMEN provides a far superior goodness of fit to the data than alternative approaches.\footnote{The AMEN approach has already been developed into a package named \pkg{amen} and is available on \href{https://cran.r-project.org/web/packages/amen/index.html}{CRAN} \citep{hoff:etal:2015}.} Further, through the AMEN approach we can estimate many different types of cross-sectional and longitudinal relational data structures (e.g., binomial, gaussian, and ordinal edges) in a straightforward way. The rest of this article proceeds as follows, we briefly motivate the need for network oriented approaches, introduce the AMEN modeling framework, compare it to previous implementations of latent variable approaches, and then end by showing how this approach fits the application presented in \citet{cranmer:etal:2016}. 

We believe that this modeling framework can provide a flexible and easy to use scheme through which scholars can study relational data. It addresses the issue of interdependence while still allowing scholars to examine theories that may only be relevant in the monadic or dyadic level. Further at the network level it provides estimates of degree related effects, reciprocity, and provides a descriptive visualization of higher order dependencies such as homophily and stochastic equivalence. 

% \footnote{We are aware of the emerging critique from \citet{jones:etal:2016} that latent variable models do not reduce the possibility for inferential error. However, like \citet{cranmer:etal:2016} they use an earlier version of the latent variable approach that, to our knowledge, no one in political science has actually applied. Further in our replication of the application presented in Cranmer et al. we show that the approach we present here produces a far better fit of the data than alternative network approaches, while also returning parameter estimates in line with the theoretical arguments described in the original paper.}

% We agree with \citet{cranmer:etal:2016} that it does not allow for the explicit testing of parameter such as the number of two-paths in a network system, however, we would note that for the most part those parameters are of little use in the types of theories that political scientists tend to develop. 

% One approach has been to incorporate endogenous network parameters into extant modeling approaches (e.g., \citealp{fowler:2006, maoz:2009a}). Alternatively, we can employ a network based approach that accounts for and provides a way to measure interdependence. Latent space approach, exponential random graph model. 

% We eschew a long discussion of the technical details behind each of these approaches because \citet{cranmer:etal:2016} have already provided it. Our primary motivation here is to better introduce a key modeling framework that has been use in Political Science that they did not address. 