There has been growing interest in the study of political networks. Network analysis allows scholars to focus away from the individual observation toward the interrelationships among observations. This has been particularly strong in the field of international relations, but also has a long tradition in American and Comparative politics.  Many network approaches developed in descriptive fashion, and until recently most network studies have been descriptive. Inferential work with networks has been growing.   \citet{cranmer:etal:2016} surveys three, inferential approaches, including the latent space and exponential random graph models of network dependencies.  We present a new approach which presents additive and multiplicative specifications that capture 1st, 2nd, and 3rd order dependencies in network data, whether those data are binary, ordinal, or continuous. In addition this approach, called AME,  also allows the incorporation of temporal dependencies.  We develop this approach and compare it to those examined in the survey by Cranmer et alia (2016).  The AME approach is shown to be a) easy to implement, b) interpretable in a general linear model framework, c) not faced with computational difficulties, d) avoids the risk of degeneracy in network sampling, e) captures 1st, 2nd, and 3rd order network dependencies, and f) substantially outperforms multiple regression quadratic assignment procedures, exponential random graph models, and latent space approaches using based on Euclidean distance metrics. As such AME offers a straightforward way to undertake nuanced, and principled network analysis for a wide range of social science questions involving binary, categorical, and continuous data. 

245 words.