There is growing interest in the study of political networks. Network analysis allows scholars to focus away from  individual observations toward interrelationships among observations. Many network approaches developed in descriptive fashion, and until recently most network studies have been descriptive. With greater interest in networks inferential work with networks has been growing. We present a new approach that models interdependencies among observation using additive and multiplicative effects, this approach can be applied to binary, ordinal, and continuous network data. In addition this approach, called AME, provides a set of tools for inference on longitudinal networks as well.  We develop this approach and compare it to those examined in the recent survey by Cranmer et al. (2016).  The AME approach is shown to be a) easy to implement, b) interpretable in a general linear model framework, c) devoid typical computational difficulties, d) avoid the risk of degeneracy in network sampling, e) capture 1st, 2nd, and 3rd order network dependencies, and f) substantially outperform multiple regression quadratic assignment procedures, exponential random graph models, and latent space approaches using  Euclidean distance metrics. AME offers a straightforward way to undertake nuanced, principled inferential network analysis for a wide range of social science questions. 

 % MPSA abstract
 % There is growing interest in the study of political networks. Network analysis allows scholars to focus away from  individual observations toward interrelationships among observations. Many network approaches developed in descriptive fashion, and until recently most network studies have been descriptive. With greater interest in networks inferential work with networks has been growing. We present a new approach that models interdependencies among observation using additive and multiplicative effects, this approach can be applied to binary, ordinal, and continuous network data. In addition this approach, called AME, provides a set of tools for inference on longitudinal networks as well.  We develop this approach and compare it to those examined in the recent survey by Cranmer et al. (2016).  The AME approach is shown to be a) easy to implement, b) interpretable in a general linear model framework, c) devoid typical computational difficulties, d) avoid the risk of degeneracy in network sampling, e) capture 1st, 2nd, and 3rd order network dependencies, and f) substantially outperform multiple regression quadratic assignment procedures, exponential random graph models, and latent space approaches using  Euclidean distance metrics. AME offers a straightforward way to undertake nuanced, principled inferential network analysis for a wide range of social science questions. 

% MPSA overview
% We introduce the Additive and Multiplicative Effects (AME) modeling framework for studying network data and show that it provides superior performance in terms of capturing network dependencies and overall performance to existing alternatives. 