%!TEX TS-program = xelatex
\documentclass[10pt, compress]{beamer}

\usetheme[usetitleprogressbar]{m}

\usepackage[export]{adjustbox}
\usepackage{etoolbox}
\usepackage{booktabs}
\usepackage{dcolumn}
\usepackage[scale=2]{ccicons}
\usepackage{color}

% math typesetting
\usepackage{array}
\usepackage{amsmath}
\usepackage{amssymb}
\usepackage{amsthm}
\usepackage{amsfonts}
\usepackage{relsize}
\usepackage{mathtools}
\usepackage{bm}

% tables
\usepackage{tabularx}
\usepackage{booktabs}
\usepackage{multicol}
\usepackage{multirow}

% graphics stuff
\usepackage{subfig}
\usepackage{graphicx}
\usepackage[space]{grffile} % allows us to specify directories that have spaces
\usepackage{tikz}

% to change enumeration symbols begin{enumerate}[(a)]
\usepackage{enumerate}

% Add some colors
\definecolor{red1}{RGB}{253,219,199}
\definecolor{red2}{RGB}{244,165,130}
\definecolor{red3}{RGB}{178,24,43}

\definecolor{green1}{RGB}{229,245,224}
\definecolor{green2}{RGB}{161,217,155}
\definecolor{green3}{RGB}{49,163,84}

\definecolor{blue0}{RGB}{255,247,251}
\definecolor{blue1}{RGB}{222,235,247}
\definecolor{blue2}{RGB}{158,202,225}
\definecolor{blue3}{RGB}{49,130,189}
\definecolor{blue4}{RGB}{4,90,141}

\definecolor{purple1}{RGB}{191,211,230}
\definecolor{purple2}{RGB}{140,150,198}
\definecolor{purple3}{RGB}{140,107,177}

\definecolor{brown1}{RGB}{246,232,195}
\definecolor{brown2}{RGB}{223,194,125}
\definecolor{brown3}{RGB}{191,129,45}

% square bracket matrices
\let\bbordermatrix\bordermatrix
\patchcmd{\bbordermatrix}{8.75}{4.75}{}{}
\patchcmd{\bbordermatrix}{\left(}{\left[}{}{}
\patchcmd{\bbordermatrix}{\right)}{\right]}{}{}

% easy command for boldface math symbols
\newcommand{\mbs}[1]{\boldsymbol{#1}}

% command for R package font
\newcommand{\pkg}[1]{{\fontseries{b}\selectfont #1}}

% approx iid
\newcommand\simiid{\stackrel{\mathclap{\normalfont\mbox{\tiny{iid}}}}{\sim}}

% references to graphics
\makeatletter
\def\input@path{{/Users/janus829/Dropbox/Research/netModels/summResults/}, {/Users/s7m/Dropbox/Research/netModels/summResults/}, {/Users/mdw/Dropbox/Research/Ongoing/netModels/summResults/}}
\graphicspath{{/Users/janus829/Dropbox/Research/netModels/summResults/}, {/Users/s7m/Dropbox/Research/netModels/summResults/},{/Users/mdw/Dropbox/research/Ongoing/netModels/summResults/}}

\title[AMEN]{\textsc{Additive and Multiplicative Latent Factor Models for Network Inference}}
\author[Minhas, Hoff, \& Ward]{Shahryar Minhas$^\dag$, Peter D. Hoff$^\ddag$, \& Michael D. Ward$^\dag$ \\ Duke University \\ $^\dag$ Department of Political Science \\ $^\ddag$ Department of Statistical Science} 

\date{\today}

\begin{document}
\frame{\titlepage}
%%%%%%%%%%%%%%%%%%%%%%%%%%%%%%%%%%%%%%%%%%%%%%%%%%%%%%%%%%%%
\frame{\frametitle{Networks are important; How to Study}
Obligatory Network Graph Here. This is Sexual Network in Typical Midwest High School (Bearman, Moody, Stovel, 2004).  
  \centering
  \includegraphics[width=.8\textwidth]{chains.jpg}
}
\frame{
  \frametitle{Promise of Exponential Random Graph Models}
  \vspace{-5mm}
\begin{itemize}
  \item - Early 1970s development of pseudolikelihood estimation: Ove Frank (1971); Julian Besag (1972) proposed using a logistic regression with network characteristics as covariates. Birth of ERGM.
  \item -  
  $P(x)= exp(\Theta^T s(x)- \psi(\theta))$, where $x$ is an adjacency matrix that is a graph, $s(\theta)$ are some set of sufficient statistics for the graph, and $\psi(\theta)$ is a normalizing constant, often set to be $\log \sum_x e^{\theta^T z{x)}}$. This is often estimated via pseudolikelihood, simply by regressing $x \sim \text{logit} (z(x))$.
  \item - Maximum Likelihood is a better approach with Robbins and Monro, or the importance sampling approach of Geyer \& Thompson. More advances with Bayesian approaches are available now with MCMC (Koskinen, Robins \& Pattison).
  \item - In the 1990s, networks became more widely recognized as important and the ERGM approach was often employed to estimate models in a variety of network domains. Needle sharing communities, HIV infections, for example.
\end{itemize}
}

\frame{
  \frametitle{There is a problem with ERGM.}
  \framesubtitle{It is a bug, not a feature.}
\vspace{-5mm}
\begin{itemize}
\item $\bullet$ Schweinberger, M. (2011). Instability, sensitivity, and degeneracy of discrete exponential families. \textbf{Journal of the American Statistical Association}, 106(496):1361–1370.
\item $\bullet$Schweinberger, M. and Handcock, M. S. (2015). Local dependence in random graph models: Characterization, properties and statistical inference. \textbf{Journal of the Royal Statistical Society: Series B (Statistical Methodology)}, 77(3):647–676.
\item $\bullet$Chatterjee, S. and Diaconis, P. (2013). Estimating and understanding exponential random graph models. \textbf{The Annals of Statistics}, 41(5):2428–2461.
\item $\bullet$Rastelli, R., Friel, N., and Raftery, A. E. (2016). Properties of latent variable network models. \textbf{Network Science}, 4(4):1–26.
\end{itemize}
}

\frame{
  \frametitle{It is a bug, not a feature.}
\vspace{-5mm}
\begin{itemize}
\item $\bullet$Probabilistic ERGM models place almost all of the probability on networks that
are either nearly empty (degenerate) with no linkages or nearly saturated with
all nodes being interconnected. 
\item $\bullet$The likelihood surface contains steep or discontinuous gradients
that render it impossible to solve numerically (or analytically).  Even (especially) for very small networks this is problematic.
\end{itemize}
\includegraphics[scale=.3]{disconuity.png}
}


%%%%%%%%%%%%%%%%%%%%%%%%%%%%%%%%%%%%%%%%%%%%%%%%%%%%%%%%%%%%
\frame{
  \frametitle{Let's (re)start with the data and build up an approach}

  \vspace{-10mm}
  Relational data consists of 
  \begin{itemize}
    \item - a set of units or nodes
    \item - a set of measurements, $y_{ij}$, specific to pairs of nodes $(i,j)$ 
  \end{itemize}

  \centering
  \includegraphics[width=1.05\textwidth]{df_adj_net3}
}
%%%%%%%%%%%%%%%%%%%%%%%%%%%%%%%%%%%%%%%%%%%%%%%%%%%%%%%%%%%%

%%%%%%%%%%%%%%%%%%%%%%%%%%%%%%%%%%%%%%%%%%%%%%%%%%%%%%%%%%%%
\frame{
  \frametitle{Relational data assumptions}

GLM: $y_{ij} \sim \beta^{T} X_{ij} + e_{ij}$

Networks typically show evidence against independence of {$e_{ij} : i \neq j$}

Not accounting for dependence can lead to:

\begin{itemize}
\item - biased effects estimation
\item - uncalibrated confidence intervals
\item - poor predictive performance
\item - inaccurate description of network phenomena
\end{itemize}

We've been hearing this concern for decades now:

\begin{tabular}{lll}
Thompson \& Walker (1982) & Beck et al. (1998) & Franzese \& Hays (2007) \\
Frank \& Strauss (1986) & Signorino (1999) &  Snijders (2011) \\
Kenny (1996) & Li \& Loken (2002) & Aronow et al. (2015) \\
Krackhardt (1998) & Hoff \& Ward (2004) & Athey et al. (2016) \\
\end{tabular}

}
%%%%%%%%%%%%%%%%%%%%%%%%%%%%%%%%%%%%%%%%%%%%%%%%%%%%%%%%%%%%

%%%%%%%%%%%%%%%%%%%%%%%%%%%%%%%%%%%%%%%%%%%%%%%%%%%%%%%%%%%%
\frame{
  \frametitle{Outline}

\begin{itemize}
  \item - Nodal and dyadic dependencies in networks
  \item \qquad -  Can model using the ``A'' in AME 
  \item - Third order dependencies
  \item \qquad - Can model using the ``M'' in AME
  \item - Application
\end{itemize}

}
%%%%%%%%%%%%%%%%%%%%%%%%%%%%%%%%%%%%%%%%%%%%%%%%%%%%%%%%%%%%

%%%%%%%%%%%%%%%%%%%%%%%%%%%%%%%%%%%%%%%%%%%%%%%%%%%%%%%%%%%%
\frame{
  \frametitle{What network phenomena? Sender heterogeneity}

  Values across a row, say $\{y_{ij},y_{ik},y_{il}\}$, may be more similar to each other than other values in the adjacency matrix because each of these values has a common sender $i$

  \centering
  \includegraphics[width=.7\textwidth]{adjRowDep}

}
%%%%%%%%%%%%%%%%%%%%%%%%%%%%%%%%%%%%%%%%%%%%%%%%%%%%%%%%%%%%

%%%%%%%%%%%%%%%%%%%%%%%%%%%%%%%%%%%%%%%%%%%%%%%%%%%%%%%%%%%%
\frame{
  \frametitle{What network phenomena? Receiver heterogeneity}

  Values across a column, say $\{y_{ji},y_{ki},y_{li}\}$, may be more similar to each other than other values in the adjacency matrix because each of these values has a common receiver $i$

  \centering
  \includegraphics[width=.7\textwidth]{adjColDep}

}
%%%%%%%%%%%%%%%%%%%%%%%%%%%%%%%%%%%%%%%%%%%%%%%%%%%%%%%%%%%%

%%%%%%%%%%%%%%%%%%%%%%%%%%%%%%%%%%%%%%%%%%%%%%%%%%%%%%%%%%%%
\frame{
  \frametitle{What network phenomena? Sender-Receiver Covariance}

  Actors who are more likely to send ties in a network may also be more likely to receive them

  \centering
  \includegraphics[width=.7\textwidth]{adjRowColCovar}

}
%%%%%%%%%%%%%%%%%%%%%%%%%%%%%%%%%%%%%%%%%%%%%%%%%%%%%%%%%%%%

%%%%%%%%%%%%%%%%%%%%%%%%%%%%%%%%%%%%%%%%%%%%%%%%%%%%%%%%%%%%
\frame{
  \frametitle{What network phenomena? Reciprocity}

  Values of $y_{ij}$ and $y_{ji}$ may be statistically dependent

  \centering
  \includegraphics[width=.7\textwidth]{adjRecip}

}
%%%%%%%%%%%%%%%%%%%%%%%%%%%%%%%%%%%%%%%%%%%%%%%%%%%%%%%%%%%%

%%%%%%%%%%%%%%%%%%%%%%%%%%%%%%%%%%%%%%%%%%%%%%%%%%%%%%%%%%%%
\frame{
  \frametitle{Social Relations Model (The ``A'' in AME)}

We use this model to form the additive effects portion of AME

\begin{align*}
\begin{aligned}
      y_{ij} &= \color{red}{\mu} + \color{red}{e_{ij}} \\
      e_{ij} &= a_{i} + b_{j} + \epsilon_{ij} \\
      \{ (a_{1}, b_{1}), \ldots, (a_{n}, b_{n}) \} &\sim N(0,\Sigma_{ab}) \\ 
      \{ (\epsilon_{ij}, \epsilon_{ji}) : \; i \neq j\} &\sim N(0,\Sigma_{\epsilon}), \text{ where } \\
      \Sigma_{ab} = \begin{pmatrix} \sigma_{a}^{2} & \sigma_{ab} \\ \sigma_{ab} & \sigma_{b}^2   \end{pmatrix} \;\;\;\;\; &\Sigma_{\epsilon} = \sigma_{\epsilon}^{2} \begin{pmatrix} 1 & \rho \\ \rho & 1  \end{pmatrix}
\end{aligned}
\end{align*}

\begin{itemize}
\item - $\mu$ baseline measure of network activity
\item - $e_{ij}$ residual variation that we will use the SRM to decompose
\end{itemize}

}
%%%%%%%%%%%%%%%%%%%%%%%%%%%%%%%%%%%%%%%%%%%%%%%%%%%%%%%%%%%%

%%%%%%%%%%%%%%%%%%%%%%%%%%%%%%%%%%%%%%%%%%%%%%%%%%%%%%%%%%%%
\frame{
  \frametitle{Social Relations Model (The ``A'' in AME)}

\begin{align*}
\begin{aligned}
      y_{ij} &= \mu + e_{ij} \\
      e_{ij} &= \color{red}{a_{i} + b_{j}} + \epsilon_{ij} \\
      \color{red}{\{ (a_{1}, b_{1}), \ldots, (a_{n}, b_{n}) \}} &\sim N(0,\Sigma_{ab}) \\ 
      \{ (\epsilon_{ij}, \epsilon_{ji}) : \; i \neq j\} &\sim N(0,\Sigma_{\epsilon}), \text{ where } \\
      \Sigma_{ab} = \begin{pmatrix} \sigma_{a}^{2} & \sigma_{ab} \\ \sigma_{ab} & \sigma_{b}^2   \end{pmatrix} \;\;\;\;\; &\Sigma_{\epsilon} = \sigma_{\epsilon}^{2} \begin{pmatrix} 1 & \rho \\ \rho & 1  \end{pmatrix}
\end{aligned}
\end{align*}

\begin{itemize}
\item - row/sender effect ($a_{i}$) \& column/receiver effect ($b_{j}$)
\item - Modeled jointly to account for correlation in how active an actor is in sending and receiving ties
\end{itemize}

}
%%%%%%%%%%%%%%%%%%%%%%%%%%%%%%%%%%%%%%%%%%%%%%%%%%%%%%%%%%%%

%%%%%%%%%%%%%%%%%%%%%%%%%%%%%%%%%%%%%%%%%%%%%%%%%%%%%%%%%%%%
\frame{
  \frametitle{Social Relations Model (The ``A'' in AME)}

\begin{align*}
\begin{aligned}
      y_{ij} &= \mu + e_{ij} \\
      e_{ij} &= a_{i} + b_{j} + \epsilon_{ij} \\
      \{ (a_{1}, b_{1}), \ldots, (a_{n}, b_{n}) \} &\sim N(0,\color{red}{\Sigma_{ab}}) \\ 
      \{ (\epsilon_{ij}, \epsilon_{ji}) : \; i \neq j\} &\sim N(0,\Sigma_{\epsilon}), \text{ where } \\
      \color{red}{\Sigma_{ab}} = \begin{pmatrix} \sigma_{a}^{2} & \sigma_{ab} \\ \sigma_{ab} & \sigma_{b}^2   \end{pmatrix} \;\;\;\;\; &\Sigma_{\epsilon} = \sigma_{\epsilon}^{2} \begin{pmatrix} 1 & \rho \\ \rho & 1  \end{pmatrix}
\end{aligned}
\end{align*}

\begin{itemize}
\item - $\sigma_{a}^{2}$ and $\sigma_{b}^{2}$ capture heterogeneity in the row and column means
\item - $\sigma_{ab}$ describes the linear relationship between these two effects (i.e., whether actors who send [receive] a lot of ties also receive [send] a lot of ties)
\end{itemize}

}
%%%%%%%%%%%%%%%%%%%%%%%%%%%%%%%%%%%%%%%%%%%%%%%%%%%%%%%%%%%%

%%%%%%%%%%%%%%%%%%%%%%%%%%%%%%%%%%%%%%%%%%%%%%%%%%%%%%%%%%%%
\frame{
  \frametitle{Social Relations Model (The ``A'' in AME)}

\begin{align*}
\begin{aligned}
      y_{ij} &= \mu + e_{ij} \\
      e_{ij} &= a_{i} + b_{j} + \color{red}{\epsilon_{ij}} \\
      \{ (a_{1}, b_{1}), \ldots, (a_{n}, b_{n}) \} &\sim N(0,\Sigma_{ab}) \\ 
      \color{red}{\{ (\epsilon_{ij}, \epsilon_{ji}) : \; i \neq j\}} &\sim N(0,\color{red}{\Sigma_{\epsilon}}), \text{ where } \\
      \Sigma_{ab} = \begin{pmatrix} \sigma_{a}^{2} & \sigma_{ab} \\ \sigma_{ab} & \sigma_{b}^2   \end{pmatrix} \;\;\;\;\; & \color{red}{\Sigma_{\epsilon}} = \sigma_{\epsilon}^{2} \begin{pmatrix} 1 & \rho \\ \rho & 1  \end{pmatrix}
\end{aligned}
\end{align*}

\begin{itemize}
\item - $\epsilon_{ij}$ captures the within dyad effect
\item - Second-order dependencies are described by $\sigma_{\epsilon}^{2}$
\item - Reciprocity, aka within dyad correlation, represented by $\rho$
\end{itemize}
}
%%%%%%%%%%%%%%%%%%%%%%%%%%%%%%%%%%%%%%%%%%%%%%%%%%%%%%%%%%%%

%%%%%%%%%%%%%%%%%%%%%%%%%%%%%%%%%%%%%%%%%%%%%%%%%%%%%%%%%%%%
\frame{
  \frametitle{Third Order Dependencies}

  \vspace{-10mm}

  \begin{table}[ht]
  \begin{tabular}{lcr}
  \scshape{Homophily} & & \scshape{Stochastic Equivalence} \\
  \includegraphics[width=.33\textwidth]{homophNet} & \hspace{2cm} &
  \includegraphics[width=.33\textwidth]{stochEquivNet}  
  \end{tabular}
  \end{table}

  To account for these patterns we can build on what we have so far and find an expression for $\gamma$:
  
  \vspace{-5mm}
  \begin{align*}
  \centering
  y_{ij} &\approx \beta^{T} X_{ij} + a_{i} + b_{j} + \gamma(u_{i},v_{j})
  \end{align*}

}
%%%%%%%%%%%%%%%%%%%%%%%%%%%%%%%%%%%%%%%%%%%%%%%%%%%%%%%%%%%%

%%%%%%%%%%%%%%%%%%%%%%%%%%%%%%%%%%%%%%%%%%%%%%%%%%%%%%%%%%%%
\frame{
  \frametitle{Latent Factor Model: The ``M'' in AME}

Each node $i$ has an unknown latent factor

\vspace{-5mm}
\begin{align*}
{\textbf{u}_{i},\textbf{v}_{j}} \in \mathbb{R}^{k} \;\; i,j \in \{1, \ldots, n \} \\
\end{align*}

\vspace{-5mm}
The probability of a tie from $i$ to $j$ depends on their latent factors

\vspace{-5mm}
\begin{align*}
\begin{aligned}
  \gamma(\textbf{u}_{i}, \textbf{v}_{j}) &= \textbf{u}_{i}^{T} D \textbf{v}_{j} \\
  &= \sum_{k \in K} d_{k} u_{ik} v_{jk} \\
  &D \text{ is a  } K \times K \text{ diagonal matrix}
\end{aligned}
\end{align*}

Can account for both stochastic equivalence and homophily.

}

\frame{
  \frametitle{Inner Product versus Euclidean Distance}

We focus on two approaches to the latent space:  the latent distance model
(LDM) and the latent factor model (LFM). 

\begin{align}
\begin{aligned}
\text{Latent distance model} \\
  &\alpha(\textbf{u}_{i}, \textbf{u}_{j}) = -|\textbf{u}_{i} - \textbf{u}_{j}| \\
  &\textbf{u}_{i} \in \mathbb{R}^{K}, \; i \in \{1, \ldots, n \} \\
\text{Latent factor model} \\
  &\alpha(\textbf{u}_{i}, \textbf{u}_{j}) = \textbf{u}_{i}^{\top} \Lambda \textbf{u}_{j} \\
  &\textbf{u}_{i} \in \mathbb{R}^{K}, \; i \in \{1, \ldots, n \} \\
  &\Lambda \text{ a } K \times K \text{ diagonal matrix}
\label{eqn:latAlpha}
\end{aligned}
\end{align}
}

%%%%%%%%%%%%%%%%%%%%%%%%%%%%%%%%%%%%%%%%%%%%%%%%%%%%%%%%%%%%
\frame{\frametitle{Putting it all together}
The AME approach can be restated as simple (\textit{simple}) regression
\begin{align}
\begin{aligned}
  y_{ij} &= g(\theta_{ij}) \\
  &\theta_{ij} = \bm\beta^{\top} \mathbf{X}_{ij} + e_{ij} \\
  &e_{ij} = a_{i} + b_{j}  + \epsilon_{ij} + \alpha(\textbf{u}_{i}, \textbf{v}_{j}) \text{  , where } \\
  &\qquad \alpha(\textbf{u}_{i}, \textbf{v}_{j}) = \textbf{u}_{i}^{\top} \textbf{D} \textbf{v}_{j} = \sum_{k \in K} d_{k} u_{ik} v_{jk}. \\
\label{eqn:ame}
\end{aligned}
\end{align}

The \texttt{amen} package implements this. Let's use it to hit some nails.
}

%%%%%%%%%%%%%%%%%%%%%%%%%%%%%%%%%%%%%%%%%%%%%%%%%%%%%%%%%%%%
\frame{
  \frametitle{Swiss Climate Change Application}

Cross-sectional network measuring whether an actor indicated that they collaborated with another during the policy design of the Swiss CO$_{2}$ act (Ingold 2008)

\begin{figure}[ht]
  \centering
  \begin{tabular}{cc}
  \includegraphics[width=.47\textwidth]{dvNet_outDegree} & 
  \includegraphics[width=.44\textwidth]{dvNet_inDegree}
  \end{tabular}
\end{figure}

}
%%%%%%%%%%%%%%%%%%%%%%%%%%%%%%%%%%%%%%%%%%%%%%%%%%%%%%%%%%%%

%%%%%%%%%%%%%%%%%%%%%%%%%%%%%%%%%%%%%%%%%%%%%%%%%%%%%%%%%%%%
\frame{
  \frametitle{Parameter Estimates}

% latex table generated in R 3.3.1 by xtable 1.8-2 package
% Sun Aug 21 03:32:43 2016
\vspace{-15mm}
\begin{table}[ht]
\centering
\begingroup\scriptsize
\begin{tabular}{lcccccc}
   & Expected & \multirow{2}{*}{Logit} & \multirow{2}{*}{MRQAP} & \multirow{2}{*}{LDM} & \multirow{2}{*}{ERGM} & \multirow{2}{*}{AME} \\ 
   & Effect & & & & & \\
  \hline
  \textbf{Conflicting policy preferences} &  &  &  &  &  &  \\ 
  $\;\;\;\;$ Business vs. NGO & \color{red}{$-$} & -0.86 & \color{red}{-0.87$^{\ast}$} & \color{red}{-1.37$^{\ast}$} & \color{red}{-1.11$^{\ast}$} & \color{red}{-1.37$^{\ast}$} \\ 
  $\;\;\;\;$ Opposition/alliance & \color{blue}{$+$} &  \color{blue}{1.21$^{\ast}$} & \color{blue}{1.14$^{\ast}$} & 0.00 & \color{blue}{1.22$^{\ast}$} & \color{blue}{1.08$^{\ast}$} \\ 
  $\;\;\;\;$ Preference dissimilarity & \color{red}{$-$} & -0.07 & -0.60 & \color{red}{-1.76$^{\ast}$} & -0.44 & \color{red}{-0.79$^{\ast}$} \\ 
  \textbf{Transaction costs} &  &  &  &  &  &  \\ 
  $\;\;\;\;$ Joint forum participation & \color{blue}{$+$} & \color{blue}{0.88$^{\ast}$} & \color{blue}{0.75$^{\ast}$} & \color{blue}{1.51$^{\ast}$} & \color{blue}{0.90$^{\ast}$} & \color{blue}{0.92$^{\ast}$} \\ 
  \textbf{Influence} & &  &  &  &  &  \\ 
  $\;\;\;\;$ Influence attribution & \color{blue}{$+$} & \color{blue}{1.20$^{\ast}$} & \color{blue}{1.29$^{\ast}$} & 0.08 & \color{blue}{1.00$^{\ast}$} & \color{blue}{1.09$^{\ast}$} \\ 
  $\;\;\;\;$ Alter's influence indegree & \color{blue}{$+$} & \color{blue}{0.10$^{\ast}$} & \color{blue}{0.11$^{\ast}$} & 0.01 & \color{blue}{0.21$^{\ast}$} & \color{blue}{0.11$^{\ast}$} \\ 
  $\;\;\;\;$ Influence absolute diff. & \color{red}{$-$} & \color{red}{-0.03$^{\ast}$} & \color{red}{-0.06$^{\ast}$} & 0.04 & \color{red}{-0.05$^{\ast}$} & \color{red}{-0.07$^{\ast}$} \\ 
  $\;\;\;\;$ Alter = Government actor & \color{blue}{$+$} & \color{blue}{0.63$^{\ast}$} & 0.68 & -0.46 & \color{blue}{1.04$^{\ast}$} & 0.55 \\ 
  \textbf{Functional requirements} &  &  &  &  &  & \\ 
  $\;\;\;\;$ Ego = Environmental NGO & \color{blue}{$+$} & \color{blue}{0.88$^{\ast}$} & 0.99 & -0.60 & \color{blue}{0.79$^{\ast}$} & 0.67 \\ 
  $\;\;\;\;$ Same actor type & \color{blue}{$+$} & \color{blue}{0.74$^{\ast}$} & \color{blue}{1.12$^{\ast}$} & \color{blue}{1.17$^{\ast}$} & \color{blue}{0.99$^{\ast}$} & \color{blue}{1.04$^{\ast}$} \\ 
   \hline
\end{tabular}
\endgroup
\end{table}

}

%%%%%%%%%%%%%%%%%%%%%%%%%%%%%%%%%%%%%%%%%%%%%%%%%%%%%%%%%%%%

%%%%%%%%%%%%%%%%%%%%%%%%%%%%%%%%%%%%%%%%%%%%%%%%%%%%%%%%%%%%
\frame{
  \frametitle{Latent Factor Visualization}

  \centering
  \vspace{-10mm}
  \includegraphics[width=.7\textwidth]{ameFitSR_2_UV}

}
%%%%%%%%%%%%%%%%%%%%%%%%%%%%%%%%%%%%%%%%%%%%%%%%%%%%%%%%%%%%

%%%%%%%%%%%%%%%%%%%%%%%%%%%%%%%%%%%%%%%%%%%%%%%%%%%%%%%%%%%%
\frame{
  \frametitle{Out of Sample Performance Assessment}

\begin{itemize}
  \item - Randomly divide the $n \times (n-1)$ data points into $S$ sets of roughly equal size, letting $s_{ij}$ be the set to which pair $\{ij\}$ is assigned.
  \item - For each $s \in \{1, \ldots, S\}$:
  \begin{itemize}
    \item - Obtain estimates of the model parameters conditional on $\{y_{ij} : s_{ij} \neq s\}$, the data on pairs not in set $s$.
    \item - For pairs $\{kl\}$ in set $s$, let $\hat y_{kl} = E[y_{kl} | \{y_{ij} : s_{ij} \neq s\}]$, the predicted value of $y_{kl}$ obtained using data not in set $s$.
  \end{itemize}
\end{itemize}

This procedure generates a sociomatrix of out-of-sample predictions of the observed data

}
%%%%%%%%%%%%%%%%%%%%%%%%%%%%%%%%%%%%%%%%%%%%%%%%%%%%%%%%%%%%

%%%%%%%%%%%%%%%%%%%%%%%%%%%%%%%%%%%%%%%%%%%%%%%%%%%%%%%%%%%%
\frame{
  \frametitle{Performance Comparison}

\begin{tabular}{cc}
\includegraphics[width=.5\textwidth]{roc_outSample} & 
\includegraphics[width=.5\textwidth]{rocPr_outSample} 
\end{tabular}

}
%%%%%%%%%%%%%%%%%%%%%%%%%%%%%%%%%%%%%%%%%%%%%%%%%%%%%%%%%%%%

%%%%%%%%%%%%%%%%%%%%%%%%%%%%%%%%%%%%%%%%%%%%%%%%%%%%%%%%%%%%
\frame{
  \frametitle{Network Dependencies}
  \centering
  \includegraphics[width=1\textwidth]{netPerfCoef}
}

\frame{\frametitle{Conclusion}
\begin{enumerate}
\item AME works for binary, count, and continuous relational data. 
\item AME allows longitudinal network data (i.e., tensors), wherein there can be different observations (nodes) in different time slices.  
\item AME is a regression based approach that has a numerically tractable likelihood function and it is not threatened by missing data. 
\item Is available on CRAN, with some of the more recent features to be delivered later this summer, but availble to interested beta testers.
\end{enumerate}}
%%%%%%%%%%%%%%%%%%%%%%%%%%%%%%%%%%%%%%%%%%%%%%%%%%%%%%%%%%%%
\frame{
  \frametitle{References}
\scalebox{0.9}{%
\begin{minipage}{\textwidth}
\begin{itemize}
\item $\bullet$
Hoff Peter D. (2008) Modeling homophily and stochastic equivalence in symmetric relational data in {\em Advances in Neural Information Processing Systems 20}, Processing Systems 21, eds. Platt J. C., Koller D., Singer Y., Roweis S.T. (MIT Press, Cambridge, MA, USA), pp. 657--664.

\item $\bullet$
Hoff Peter D. (2009) Multiplicative latent factor models for description and prediction of social networks. {\em Computational and Mathematical Organization Theory} 15(4):261--272.

\item $\bullet$
Hoff Peter D., Bailey Fosdick, Volfovsky Alex, Katherine Stovel. (2013) Likelihoods for fixed rank nomination networks. {\em Network Science} 1(3):253--277.

\item $\bullet$
Hoff Peter D. (2015) Multilinear tensor regression for longitudinal relational data.
 {\em The Annals of Applied Statistics} 9(3):1169--1193.

\item $\bullet$
Hoff Peter D., Fosdick Bailey, Volfovsky Alex, He Yu (2015). {\em amen: {Additive} and {Multiplicative} Effects Models for Networks and Relational Data}. R package version 1.1; 1.3 in 2017; 1.4 real soon now.
\end{itemize}
\end{minipage}
}
}
%%%%%%%%%%%%%%%%%%%%%%%%%%%%%%%%%%%%%%%%%%%%%%%%%%%%%%%%%%%%

%%%%%%%%%%%%%%%%%%%%%%%%%%%%%%%%%%%%%%%%%%%%%%%%%%%%%%%%%%%%

%%%%%%%%%%%%%%%%%%%%%%%%%%%%%%%%%%%%%%%%%%%%%%%%%%%%%%%%%%%%
\plain{Thanks.}
%%%%%%%%%%%%%%%%%%%%%%%%%%%%%%%%%%%%%%%%%%%%%%%%%%%%%%%%%%%%




%%%%%%%%%%%%%%%%%%%%%%%%%%%%%%%%%%%%%%%%%%%%%%%%%%%%%%%%%%%%
\frame{
  \frametitle{Standard Network Dependence Measures}
\vspace{-5mm}
\includegraphics[width=1\textwidth]{ggGofAll_preeze}
}


%%%%%%%%%%%%%%%%%%%%%%%%%%%%%%%%%%%%%%%%%%%%%%%%%%%%%%%%%%%%
\frame{
  \frametitle{Simulation Comparison}

\includegraphics[width=1\textwidth]{ameVergmSim.png}

}
%%%%%%%%%%%%%%%%%%%%%%%%%%%%%%%%%%%%%%%%%%%%%%%%%%%%%%%%%%%%

%%%%%%%%%%%%%%%%%%%%%%%%%%%%%%%%%%%%%%%%%%%%%%%%%%%%%%%%%%%%
\frame{
  \frametitle{AMEN v LSM Performance}

  \begin{tabular}{cc}
  \includegraphics[width=.5\textwidth]{roc_latSpace_outSample} & 
  \includegraphics[width=.5\textwidth]{rocPr_latSpace_outSample}
  \end{tabular}

}
%%%%%%%%%%%%%%%%%%%%%%%%%%%%%%%%%%%%%%%%%%%%%%%%%%%%%%%%%%%%

%%%%%%%%%%%%%%%%%%%%%%%%%%%%%%%%%%%%%%%%%%%%%%%%%%%%%%%%%%%%
\frame{
  \frametitle{AMEN versus LSM Net Dependence}

  \includegraphics[width=1\textwidth]{netPerfCoef_latSpace}
}
%%%%%%%%%%%%%%%%%%%%%%%%%%%%%%%%%%%%%%%%%%%%%%%%%%%%%%%%%%%%

%%%%%%%%%%%%%%%%%%%%%%%%%%%%%%%%%%%%%%%%%%%%%%%%%%%%%%%%%%%%
\frame{
  \frametitle{AMEN varying $K$ Performance}

  \begin{tabular}{cc}
  \includegraphics[width=.5\textwidth]{roc_ameSR_outSample} & 
  \includegraphics[width=.5\textwidth]{rocPr_ameSR_outSample}
  \end{tabular}

}
%%%%%%%%%%%%%%%%%%%%%%%%%%%%%%%%%%%%%%%%%%%%%%%%%%%%%%%%%%%%

%%%%%%%%%%%%%%%%%%%%%%%%%%%%%%%%%%%%%%%%%%%%%%%%%%%%%%%%%%%%
\frame{
  \frametitle{AMEN varying $K$ Net Dependence}

  \includegraphics[width=1\textwidth]{netPerfCoef_ameSR}
}
%%%%%%%%%%%%%%%%%%%%%%%%%%%%%%%%%%%%%%%%%%%%%%%%%%%%%%%%%%%%

\end{document}
