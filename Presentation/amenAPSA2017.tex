%!TEX TS-program = xelatex
\documentclass[10pt, compress]{beamer}

\usetheme[usetitleprogressbar]{m}

\usepackage[export]{adjustbox}
\usepackage{etoolbox}
\usepackage{booktabs}
\usepackage{dcolumn}
\usepackage[scale=2]{ccicons}
\usepackage{color}

% math typesetting
\usepackage{array}
\usepackage{amsmath}
\usepackage{amssymb}
\usepackage{amsthm}
\usepackage{amsfonts}
\usepackage{relsize}
\usepackage{mathtools}
\usepackage{bm}

% tables
\usepackage{tabularx}
\usepackage{booktabs}
\usepackage{multicol}
\usepackage{multirow}

% graphics stuff
\usepackage{subfig}
\usepackage{graphicx}
\usepackage[space]{grffile} % allows us to specify directories that have spaces
\usepackage{tikz}

% to change enumeration symbols begin{enumerate}[(a)]
\usepackage{enumerate}

% Add some colors
\definecolor{red1}{RGB}{253,219,199}
\definecolor{red2}{RGB}{244,165,130}
\definecolor{red3}{RGB}{178,24,43}

\definecolor{green1}{RGB}{229,245,224}
\definecolor{green2}{RGB}{161,217,155}
\definecolor{green3}{RGB}{49,163,84}

\definecolor{blue0}{RGB}{255,247,251}
\definecolor{blue1}{RGB}{222,235,247}
\definecolor{blue2}{RGB}{158,202,225}
\definecolor{blue3}{RGB}{49,130,189}
\definecolor{blue4}{RGB}{4,90,141}

\definecolor{purple1}{RGB}{191,211,230}
\definecolor{purple2}{RGB}{140,150,198}
\definecolor{purple3}{RGB}{140,107,177}

\definecolor{brown1}{RGB}{246,232,195}
\definecolor{brown2}{RGB}{223,194,125}
\definecolor{brown3}{RGB}{191,129,45}

% square bracket matrices
\let\bbordermatrix\bordermatrix
\patchcmd{\bbordermatrix}{8.75}{4.75}{}{}
\patchcmd{\bbordermatrix}{\left(}{\left[}{}{}
\patchcmd{\bbordermatrix}{\right)}{\right]}{}{}

% easy command for boldface math symbols
\newcommand{\mbs}[1]{\boldsymbol{#1}}

% command for R package font
\newcommand{\pkg}[1]{{\fontseries{b}\selectfont #1}}

% approx iid
\newcommand\simiid{\stackrel{\mathclap{\normalfont\mbox{\tiny{iid}}}}{\sim}}

% references to graphics
\makeatletter
\def\input@path{{/Users/janus829/Dropbox/Research/netModels/summResults/}, {/Users/s7m/Dropbox/Research/netModels/summResults/}, {/Users/mdw/Dropbox/Research/Ongoing/netModels/summResults/}}
\graphicspath{{/Users/janus829/Dropbox/Research/netModels/summResults/}, {/Users/s7m/Dropbox/Research/netModels/summResults/},{/Users/mdw/Dropbox/Research/Ongoing/netModels/summResults/}}

\title[AMEN]{\textsc{Inferential Approaches for Network Analysis: AMEN for Latent Factor Models}}
\author[Minhas, Hoff, \& Ward]
{Shahryar Minhas$^\dag$, Peter D. Hoff$^\ddag$, \& Michael D. Ward$^\gimel$ \\~\\~\\~\\ 
$^\dag$ Department of Political Science, Michigan State University\\ $^\ddag$ Department of Statistical Sciences, Duke University\\
$^\gimel$ Department of Political Science, Duke University\\~\\~} 
%\vspace[*]{5 cm}
\date{Prepared for the 113th APSA Annual Meetings of the American Political Science Association, San Francisco, August 31 – September 3, 2017. \today}



\begin{document}
\frame{\titlepage}

%%%%%%%%%%%%%%%%%%%%%%%%%%%%%%%%%%%%%%%%%%%%%%%%%%%%%%%%%%%%
\frame{
  \frametitle{Motivation}

  \vspace{-10mm}
  Relational data consists of 
  \begin{itemize}
    \item - a set of units or nodes
    \item - a set of measurements, $y_{ij}$, specific to pairs of nodes $(i,j)$ 
  \end{itemize}

  \centering
  \includegraphics[width=1.05\textwidth]{df_adj_net3}
}
%%%%%%%%%%%%%%%%%%%%%%%%%%%%%%%%%%%%%%%%%%%%%%%%%%%%%%%%%%%%

%%%%%%%%%%%%%%%%%%%%%%%%%%%%%%%%%%%%%%%%%%%%%%%%%%%%%%%%%%%%
\frame{
  \frametitle{Relational data assumptions}

GLM: $y_{ij} \sim \beta^{T} X_{ij} + e_{ij}$

Networks typically show evidence against independence of {$e_{ij} : i \neq j$}

Not accounting for dependence can lead to:

\begin{itemize}
\item - biased effects estimation
\item - uncalibrated confidence intervals
\item - poor predictive performance
\item - inaccurate description of network phenomena
\end{itemize}

We've been hearing this concern for decades now:

\begin{tabular}{lll}
Thompson \& Walker (1982) & Beck et al. (1998) & Snijders (2011) \\
Frank \& Strauss (1986) & Signorino (1999) & Erikson et al. (2014) \\
Kenny (1996) & Li \& Loken (2002) & Aronow et al. (2015) \\
Krackhardt (1998) & Hoff \& Ward (2004) & Athey et al. (2016) \\
\end{tabular}

}
%%%%%%%%%%%%%%%%%%%%%%%%%%%%%%%%%%%%%%%%%%%%%%%%%%%%%%%%%%%%


%%%%%%%%%%%%%%%%%%%%%%%%%%%%%%%%%%%%%%%%%%%%%%%%%%%%%%%%%%%%
\frame{
  \frametitle{What network phenomena? Sender heterogeneity}

  Values across a row, say $\{y_{ij},y_{ik},y_{il}\}$, may be more similar to each other than other values in the adjacency matrix because each of these values has a common sender $i$

  \centering
  \includegraphics[width=.7\textwidth]{adjRowDep}

}
%%%%%%%%%%%%%%%%%%%%%%%%%%%%%%%%%%%%%%%%%%%%%%%%%%%%%%%%%%%%

%%%%%%%%%%%%%%%%%%%%%%%%%%%%%%%%%%%%%%%%%%%%%%%%%%%%%%%%%%%%
\frame{
  \frametitle{What network phenomena? Receiver heterogeneity}

  Values across a column, say $\{y_{ji},y_{ki},y_{li}\}$, may be more similar to each other than other values in the adjacency matrix because each of these values has a common receiver $i$

  \centering
  \includegraphics[width=.7\textwidth]{adjColDep}

}
%%%%%%%%%%%%%%%%%%%%%%%%%%%%%%%%%%%%%%%%%%%%%%%%%%%%%%%%%%%%

%%%%%%%%%%%%%%%%%%%%%%%%%%%%%%%%%%%%%%%%%%%%%%%%%%%%%%%%%%%%
\frame{
  \frametitle{What network phenomena? Sender-Receiver Covariance}

  Actors who are more likely to send ties in a network may also be more likely to receive them

  \centering
  \includegraphics[width=.7\textwidth]{adjRowColCovar}

}
%%%%%%%%%%%%%%%%%%%%%%%%%%%%%%%%%%%%%%%%%%%%%%%%%%%%%%%%%%%%

%%%%%%%%%%%%%%%%%%%%%%%%%%%%%%%%%%%%%%%%%%%%%%%%%%%%%%%%%%%%
\frame{
  \frametitle{What network phenomena? Reciprocity}

  Values of $y_{ij}$ and $y_{ji}$ may be statistically dependent

  \centering
  \includegraphics[width=.7\textwidth]{adjRecip}

}
%%%%%%%%%%%%%%%%%%%%%%%%%%%%%%%%%%%%%%%%%%%%%%%%%%%%%%%%%%%%

%%%%%%%%%%%%%%%%%%%%%%%%%%%%%%%%%%%%%%%%%%%%%%%%%%%%%%%%%%%%
\frame{
  \frametitle{Social Relations Model (The ``A'' in AME)}

We use this model to form the additive effects portion of AME

\begin{align*}
\begin{aligned}
      y_{ij} &= \color{red}{\mu} + \color{red}{e_{ij}} \\
      e_{ij} &= a_{i} + b_{j} + \epsilon_{ij} \\
      \{ (a_{1}, b_{1}), \ldots, (a_{n}, b_{n}) \} &\sim N(0,\Sigma_{ab}) \\ 
      \{ (\epsilon_{ij}, \epsilon_{ji}) : \; i \neq j\} &\sim N(0,\Sigma_{\epsilon}), \text{ where } \\
      \Sigma_{ab} = \begin{pmatrix} \sigma_{a}^{2} & \sigma_{ab} \\ \sigma_{ab} & \sigma_{b}^2   \end{pmatrix} \;\;\;\;\; &\Sigma_{\epsilon} = \sigma_{\epsilon}^{2} \begin{pmatrix} 1 & \rho \\ \rho & 1  \end{pmatrix}
\end{aligned}
\end{align*}

\begin{itemize}
\item - $\mu$ baseline measure of network activity
\item - $e_{ij}$ residual variation that we will use the SRM to decompose
\end{itemize}

}
%%%%%%%%%%%%%%%%%%%%%%%%%%%%%%%%%%%%%%%%%%%%%%%%%%%%%%%%%%%%

%%%%%%%%%%%%%%%%%%%%%%%%%%%%%%%%%%%%%%%%%%%%%%%%%%%%%%%%%%%%
\frame{
  \frametitle{Social Relations Model (The ``A'' in AME)}

\begin{align*}
\begin{aligned}
      y_{ij} &= \mu + e_{ij} \\
      e_{ij} &= \color{red}{a_{i} + b_{j}} + \epsilon_{ij} \\
      \color{red}{\{ (a_{1}, b_{1}), \ldots, (a_{n}, b_{n}) \}} &\sim N(0,\Sigma_{ab}) \\ 
      \{ (\epsilon_{ij}, \epsilon_{ji}) : \; i \neq j\} &\sim N(0,\Sigma_{\epsilon}), \text{ where } \\
      \Sigma_{ab} = \begin{pmatrix} \sigma_{a}^{2} & \sigma_{ab} \\ \sigma_{ab} & \sigma_{b}^2   \end{pmatrix} \;\;\;\;\; &\Sigma_{\epsilon} = \sigma_{\epsilon}^{2} \begin{pmatrix} 1 & \rho \\ \rho & 1  \end{pmatrix}
\end{aligned}
\end{align*}

\begin{itemize}
\item - row/sender effect ($a_{i}$) \& column/receiver effect ($b_{j}$)
\item - Modeled jointly to account for correlation in how active an actor is in sending and receiving ties
\end{itemize}

}
%%%%%%%%%%%%%%%%%%%%%%%%%%%%%%%%%%%%%%%%%%%%%%%%%%%%%%%%%%%%

%%%%%%%%%%%%%%%%%%%%%%%%%%%%%%%%%%%%%%%%%%%%%%%%%%%%%%%%%%%%
\frame{
  \frametitle{Social Relations Model (The ``A'' in AME)}

\begin{align*}
\begin{aligned}
      y_{ij} &= \mu + e_{ij} \\
      e_{ij} &= a_{i} + b_{j} + \epsilon_{ij} \\
      \{ (a_{1}, b_{1}), \ldots, (a_{n}, b_{n}) \} &\sim N(0,\color{red}{\Sigma_{ab}}) \\ 
      \{ (\epsilon_{ij}, \epsilon_{ji}) : \; i \neq j\} &\sim N(0,\Sigma_{\epsilon}), \text{ where } \\
      \color{red}{\Sigma_{ab}} = \begin{pmatrix} \sigma_{a}^{2} & \sigma_{ab} \\ \sigma_{ab} & \sigma_{b}^2   \end{pmatrix} \;\;\;\;\; &\Sigma_{\epsilon} = \sigma_{\epsilon}^{2} \begin{pmatrix} 1 & \rho \\ \rho & 1  \end{pmatrix}
\end{aligned}
\end{align*}

\begin{itemize}
\item - $\sigma_{a}^{2}$ and $\sigma_{b}^{2}$ capture heterogeneity in the row and column means
\item - $\sigma_{ab}$ describes the linear relationship between these two effects (i.e., whether actors who send [receive] a lot of ties also receive [send] a lot of ties)
\end{itemize}

}
%%%%%%%%%%%%%%%%%%%%%%%%%%%%%%%%%%%%%%%%%%%%%%%%%%%%%%%%%%%%

%%%%%%%%%%%%%%%%%%%%%%%%%%%%%%%%%%%%%%%%%%%%%%%%%%%%%%%%%%%%
\frame{
  \frametitle{Social Relations Model (The ``A'' in AME)}

\begin{align*}
\begin{aligned}
      y_{ij} &= \mu + e_{ij} \\
      e_{ij} &= a_{i} + b_{j} + \color{red}{\epsilon_{ij}} \\
      \{ (a_{1}, b_{1}), \ldots, (a_{n}, b_{n}) \} &\sim N(0,\Sigma_{ab}) \\ 
      \color{red}{\{ (\epsilon_{ij}, \epsilon_{ji}) : \; i \neq j\}} &\sim N(0,\color{red}{\Sigma_{\epsilon}}), \text{ where } \\
      \Sigma_{ab} = \begin{pmatrix} \sigma_{a}^{2} & \sigma_{ab} \\ \sigma_{ab} & \sigma_{b}^2   \end{pmatrix} \;\;\;\;\; & \color{red}{\Sigma_{\epsilon}} = \sigma_{\epsilon}^{2} \begin{pmatrix} 1 & \rho \\ \rho & 1  \end{pmatrix}
\end{aligned}
\end{align*}

\begin{itemize}
\item - $\epsilon_{ij}$ captures the within dyad effect
\item - Second-order dependencies are described by $\sigma_{\epsilon}^{2}$
\item - Reciprocity, aka within dyad correlation, represented by $\rho$
\end{itemize}
}
%%%%%%%%%%%%%%%%%%%%%%%%%%%%%%%%%%%%%%%%%%%%%%%%%%%%%%%%%%%%

%%%%%%%%%%%%%%%%%%%%%%%%%%%%%%%%%%%%%%%%%%%%%%%%%%%%%%%%%%%%
\frame{
  \frametitle{Third Order Dependencies}

  \vspace{-10mm}

  \begin{table}[ht]
  \begin{tabular}{lcr}
  \scshape{Homophily} & & \scshape{Stochastic Equivalence} \\
  \includegraphics[width=.33\textwidth]{homophNet} & \hspace{2cm} &
  \includegraphics[width=.33\textwidth]{stochEquivNet}  
  \end{tabular}
  \end{table}

  To account for these patterns we can build on what we have so far and find an expression for $\gamma$:
  
  \vspace{-5mm}
  \begin{align*}
  \centering
  y_{ij} &\approx \beta^{T} X_{ij} + a_{i} + b_{j} + \gamma(u_{i},v_{j})
  \end{align*}

}
%%%%%%%%%%%%%%%%%%%%%%%%%%%%%%%%%%%%%%%%%%%%%%%%%%%%%%%%%%%%

%%%%%%%%%%%%%%%%%%%%%%%%%%%%%%%%%%%%%%%%%%%%%%%%%%%%%%%%%%%%
\frame{
  \frametitle{Latent Factor Model: The ``M'' in AME}

Each node $i$ has an unknown latent factor

\vspace{-5mm}
\begin{align*}
{\textbf{u}_{i},\textbf{v}_{j}} \in \mathbb{R}^{k} \;\; i,j \in \{1, \ldots, n \} \\
\end{align*}

\vspace{-5mm}
The probability of a tie from $i$ to $j$ depends on their latent factors

\vspace{-5mm}
\begin{align*}
\begin{aligned}
  \gamma(\textbf{u}_{i}, \textbf{v}_{j}) &= \textbf{u}_{i}^{T} D \textbf{v}_{j} \\
  &= \sum_{k \in K} d_{k} u_{ik} v_{jk} \\
  &D \text{ is a  } K \times K \text{ diagonal matrix}
\end{aligned}
\end{align*}

Can account for both stochastic equivalence and homophily

}
%%%%%%%%%%%%%%%%%%%%%%%%%%%%%%%%%%%%%%%%%%%%%%%%%%%%%%%%%%%%

%%%%%%%%%%%%%%%%%%%%%%%%%%%%%%%%%%%%%%%%%%%%%%%%%%%%%%%%%%%%
\frame{
  \frametitle{Swiss Climate Change Application}

Cross-sectional network measuring whether an actor indicated that they collaborated with another during the policy design of the Swiss CO$_{2}$ act (Ingold 2008)

\begin{figure}[ht]
  \centering
  \begin{tabular}{cc}
  \includegraphics[width=.47\textwidth]{dvNet_outDegree} & 
  \includegraphics[width=.44\textwidth]{dvNet_inDegree}
  \end{tabular}
\end{figure}

}
%%%%%%%%%%%%%%%%%%%%%%%%%%%%%%%%%%%%%%%%%%%%%%%%%%%%%%%%%%%%

%%%%%%%%%%%%%%%%%%%%%%%%%%%%%%%%%%%%%%%%%%%%%%%%%%%%%%%%%%%%
\frame{
  \frametitle{Parameter Estimates}

% latex table generated in R 3.3.1 by xtable 1.8-2 package
% Sun Aug 21 03:32:43 2016
\vspace{-15mm}
\begin{table}[ht]
\centering
\begingroup\scriptsize
\begin{tabular}{lcccccc}
   & Expected & \multirow{2}{*}{Logit} & \multirow{2}{*}{MRQAP} & \multirow{2}{*}{LSM} & \multirow{2}{*}{ERGM} & \multirow{2}{*}{AME} \\ 
   & Effect & & & & & \\
  \hline
  \textbf{Conflicting policy preferences} &  &  &  &  &  &  \\ 
  $\;\;\;\;$ Business vs. NGO & \color{red}{$-$} & -0.86 & \color{red}{-0.87$^{\ast}$} & \color{red}{-1.37$^{\ast}$} & \color{red}{-1.11$^{\ast}$} & \color{red}{-1.37$^{\ast}$} \\ 
  $\;\;\;\;$ Opposition/alliance & \color{blue}{$+$} &  \color{blue}{1.21$^{\ast}$} & \color{blue}{1.14$^{\ast}$} & 0.00 & \color{blue}{1.22$^{\ast}$} & \color{blue}{1.08$^{\ast}$} \\ 
  $\;\;\;\;$ Preference dissimilarity & \color{red}{$-$} & -0.07 & -0.60 & \color{red}{-1.76$^{\ast}$} & -0.44 & \color{red}{-0.79$^{\ast}$} \\ 
  \textbf{Transaction costs} &  &  &  &  &  &  \\ 
  $\;\;\;\;$ Joint forum participation & \color{blue}{$+$} & \color{blue}{0.88$^{\ast}$} & \color{blue}{0.75$^{\ast}$} & \color{blue}{1.51$^{\ast}$} & \color{blue}{0.90$^{\ast}$} & \color{blue}{0.92$^{\ast}$} \\ 
  \textbf{Influence} & &  &  &  &  &  \\ 
  $\;\;\;\;$ Influence attribution & \color{blue}{$+$} & \color{blue}{1.20$^{\ast}$} & \color{blue}{1.29$^{\ast}$} & 0.08 & \color{blue}{1.00$^{\ast}$} & \color{blue}{1.09$^{\ast}$} \\ 
  $\;\;\;\;$ Alter's influence indegree & \color{blue}{$+$} & \color{blue}{0.10$^{\ast}$} & \color{blue}{0.11$^{\ast}$} & 0.01 & \color{blue}{0.21$^{\ast}$} & \color{blue}{0.11$^{\ast}$} \\ 
  $\;\;\;\;$ Influence absolute diff. & \color{red}{$-$} & \color{red}{-0.03$^{\ast}$} & \color{red}{-0.06$^{\ast}$} & 0.04 & \color{red}{-0.05$^{\ast}$} & \color{red}{-0.07$^{\ast}$} \\ 
  $\;\;\;\;$ Alter = Government actor & \color{blue}{$+$} & \color{blue}{0.63$^{\ast}$} & 0.68 & -0.46 & \color{blue}{1.04$^{\ast}$} & 0.55 \\ 
  \textbf{Functional requirements} &  &  &  &  &  & \\ 
  $\;\;\;\;$ Ego = Environmental NGO & \color{blue}{$+$} & \color{blue}{0.88$^{\ast}$} & 0.99 & -0.60 & \color{blue}{0.79$^{\ast}$} & 0.67 \\ 
  $\;\;\;\;$ Same actor type & \color{blue}{$+$} & \color{blue}{0.74$^{\ast}$} & \color{blue}{1.12$^{\ast}$} & \color{blue}{1.17$^{\ast}$} & \color{blue}{0.99$^{\ast}$} & \color{blue}{1.04$^{\ast}$} \\ 
   \hline
\end{tabular}
\endgroup
\end{table}

}
%%%%%%%%%%%%%%%%%%%%%%%%%%%%%%%%%%%%%%%%%%%%%%%%%%%%%%%%%%%%

%%%%%%%%%%%%%%%%%%%%%%%%%%%%%%%%%%%%%%%%%%%%%%%%%%%%%%%%%%%%
\frame{
  \frametitle{Latent Factor Visualization}

  \centering
  \vspace{-10mm}
  \includegraphics[width=.7\textwidth]{ameFitSR_2_UV}

}
%%%%%%%%%%%%%%%%%%%%%%%%%%%%%%%%%%%%%%%%%%%%%%%%%%%%%%%%%%%%

%%%%%%%%%%%%%%%%%%%%%%%%%%%%%%%%%%%%%%%%%%%%%%%%%%%%%%%%%%%%
\frame{
  \frametitle{Out of Sample Performance Assessment}

\begin{itemize}
  \item - Randomly divide the $n \times (n-1)$ data points into $S$ sets of roughly equal size, letting $s_{ij}$ be the set to which pair $\{ij\}$ is assigned.
  \item - For each $s \in \{1, \ldots, S\}$:
  \begin{itemize}
    \item - Obtain estimates of the model parameters conditional on $\{y_{ij} : s_{ij} \neq s\}$, the data on pairs not in set $s$.
    \item - For pairs $\{kl\}$ in set $s$, let $\hat y_{kl} = E[y_{kl} | \{y_{ij} : s_{ij} \neq s\}]$, the predicted value of $y_{kl}$ obtained using data not in set $s$.
  \end{itemize}
\end{itemize}

This procedure generates a sociomatrix of out-of-sample predictions of the observed data

}
%%%%%%%%%%%%%%%%%%%%%%%%%%%%%%%%%%%%%%%%%%%%%%%%%%%%%%%%%%%%

%%%%%%%%%%%%%%%%%%%%%%%%%%%%%%%%%%%%%%%%%%%%%%%%%%%%%%%%%%%%
\frame{
  \frametitle{Performance Comparison}

\begin{tabular}{cc}
\includegraphics[width=.5\textwidth]{roc_outSample} & 
\includegraphics[width=.5\textwidth]{rocPr_outSample} 
\end{tabular}

}
%%%%%%%%%%%%%%%%%%%%%%%%%%%%%%%%%%%%%%%%%%%%%%%%%%%%%%%%%%%%

%%%%%%%%%%%%%%%%%%%%%%%%%%%%%%%%%%%%%%%%%%%%%%%%%%%%%%%%%%%%
\frame{
  \frametitle{Network Dependencies}


  \centering
  \includegraphics[width=1\textwidth]{netPerfCoef}

}
%%%%%%%%%%%%%%%%%%%%%%%%%%%%%%%%%%%%%%%%%%%%%%%%%%%%%%%%%%%%

%%%%%%%%%%%%%%%%%%%%%%%%%%%%%%%%%%%%%%%%%%%%%%%%%%%%%%%%%%%%

%%%%%%%%%%%%%%%%%%%%%%%%%%%%%%%%%%%%%%%%%%%%%%%%%%%%%%%%%%%%

%%%%%%%%%%%%%%%%%%%%%%%%%%%%%%%%%%%%%%%%%%%%%%%%%%%%%%%%%%%%
\plain{Thanks.}
%%%%%%%%%%%%%%%%%%%%%%%%%%%%%%%%%%%%%%%%%%%%%%%%%%%%%%%%%%%%

%%%%%%%%%%%%%%%%%%%%%%%%%%%%%%%%%%%%%%%%%%%%%%%%%%%%%%%%%%%%
\frame{
  \frametitle{There is a problem with ERGM.}
  \framesubtitle{It is a bug, not a feature.}
\vspace{-5mm}
It is a bug, not a feature. If you don't believe us, believe these scholars:
\begin{enumerate}
\item Schweinberger, M. (2011). Instability, sensitivity, and degeneracy of discrete expo- nential families. \textbf{Journal of the American Statistical Association}, 106(496):1361–1370.
\item Schweinberger, M. and Handcock, M. S. (2015). Local dependence in random graph models: Characterization, properties and statistical inference. \textbf{Journal of the Royal Statistical Society: Series B (Statistical Methodology)}, 77(3):647–676.
\item Chatterjee, S. and Diaconis, P. (2013). Estimating and understanding ex- ponential random graph models. \textbf{The Annals of Statistics}, 41(5):2428–2461.
\item Rastelli, R., Friel, N., and Raftery, A. E. (2016). Properties of latent variable network models. \textbf{Network Science}, 4(4):1–26.
\end{enumerate}
}
%%%%%%%%%%%%%%%%%%%%%%%%%%%%%%%%%%%%%%%%%%%%%%%%%%%%%%%%%%%%

%%%%%%%%%%%%%%%%%%%%%%%%%%%%%%%%%%%%%%%%%%%%%%%%%%%%%%%%%%%%
\frame{
  \frametitle{What Is Really Wrong? }
\vspace{-5mm}

You may assume this is your likelihood surface.  Courtesy of Chatterjee and Diaconis, who concluded that sufficient statistics are not sufficient to solve this problem.  Estimation of ERGMs contains this flaw, with pseudolikelihood, maximum likelihood, or MCMC. 
\includegraphics[width=1\textwidth]{disconuity.png}
}
%%%%%%%%%%%%%%%%%%%%%%%%%%%%%%%%%%%%%%%%%%%%%%%%%%%%%%%%%%%%

%%%%%%%%%%%%%%%%%%%%%%%%%%%%%%%%%%%%%%%%%%%%%%%%%%%%%%%%%%%%
\frame{
  \frametitle{What Does This Mean? }
\vspace{-5mm}
\begin{enumerate}
\item Probabilistic ERGM models place almost all of the probability on networks that
are either nearly empty (degenerate) with no linkages or nearly saturated with
all nodes being interconnected. 
\item The likelihood surface contains steep or discontinuous gradients
that render it impossible to solve numerically (or analytically).  Even (especially) for very small networks this is problematic.
\item Find a different way to estimate network data. (See above!). 
\end{enumerate}
}
%%%%%%%%%%%%%%%%%%%%%%%%%%%%%%%%%%%%%%%%%%%%%%%%%%%%%%%%%%%%

%%%%%%%%%%%%%%%%%%%%%%%%%%%%%%%%%%%%%%%%%%%%%%%%%%%%%%%%%%%%
\frame{
  \frametitle{Extant Political Science Examples}
\vspace{-5mm}
Extant Political Science Examples Using Latent Distance Models Are Wrong.
\begin{enumerate}
\item There is a problem with the MCMC estimator in \texttt{latentnet}. Don't use it until it is fixed.
\item Latent distance models are hard to correctly interpret, in prominent evaluations in the literature it is done incorrectly.  This is explained in their introduction a decade ago.
\item The advice in Cranmer et al is wrong: Latent network models are not biased and when correctly implemented out perform ERGM models.
\end{enumerate}
}
%%%%%%%%%%%%%%%%%%%%%%%%%%%%%%%%%%%%%%%%%%%%%%%%%%%%%%%%%%%%

%%%%%%%%%%%%%%%%%%%%%%%%%%%%%%%%%%%%%%%%%%%%%%%%%%%%%%%%%%%%
\frame{
  \frametitle{Much current advice is wrong.}
  \textbf{Unfortunately some misleading and incorrect advice is in the current literature. Maybe it will be corrected.}

\begin{enumerate}
\item Cranmer, Leifeld, McClurg, Rolfe (2017 AJPS) stated that latent space models are only preferable if there are few isolates, enough observations, and the interdependence is theoretically uninteresting. 

\item All of this is wrong. Not only do they incorrectly assume all latent models to be based on Euclidean distance but their statistical comparisons are based on a program that contains known errors. Moreover, using BIC to choose among different network models is itself questionable.

\item It is not true that ``ERGM offers the most preferable combination of model fit and number of parameters,'' despite their assertions.
\end{enumerate}
}
%%%%%%%%%%%%%%%%%%%%%%%%%%%%%%%%%%%%%%%%%%%%%%%%%%%%%%%%%%%%



%%%%%%%%%%%%%%%%%%%%%%%%%%%%%%%%%%%%%%%%%%%%%%%%%%%%%%%%%%%%
\frame{
  \frametitle{Simulation Comparison}

\includegraphics[width=1\textwidth]{ameVergmSim.png}

}
%%%%%%%%%%%%%%%%%%%%%%%%%%%%%%%%%%%%%%%%%%%%%%%%%%%%%%%%%%%%

%%%%%%%%%%%%%%%%%%%%%%%%%%%%%%%%%%%%%%%%%%%%%%%%%%%%%%%%%%%%
\frame{
  \frametitle{AMEN v LSM Performance}

  \begin{tabular}{cc}
  \includegraphics[width=.5\textwidth]{roc_latSpace_outSample} & 
  \includegraphics[width=.5\textwidth]{rocPr_latSpace_outSample}
  \end{tabular}

}
%%%%%%%%%%%%%%%%%%%%%%%%%%%%%%%%%%%%%%%%%%%%%%%%%%%%%%%%%%%%

%%%%%%%%%%%%%%%%%%%%%%%%%%%%%%%%%%%%%%%%%%%%%%%%%%%%%%%%%%%%
\frame{
  \frametitle{AMEN versus LSM Net Dependence}

  \includegraphics[width=1\textwidth]{netPerfCoef_latSpace}
}
%%%%%%%%%%%%%%%%%%%%%%%%%%%%%%%%%%%%%%%%%%%%%%%%%%%%%%%%%%%%

%%%%%%%%%%%%%%%%%%%%%%%%%%%%%%%%%%%%%%%%%%%%%%%%%%%%%%%%%%%%
\frame{
  \frametitle{AMEN varying $K$ Performance}

  \begin{tabular}{cc}
  \includegraphics[width=.5\textwidth]{roc_ameSR_outSample} & 
  \includegraphics[width=.5\textwidth]{rocPr_ameSR_outSample}
  \end{tabular}

}
%%%%%%%%%%%%%%%%%%%%%%%%%%%%%%%%%%%%%%%%%%%%%%%%%%%%%%%%%%%%

%%%%%%%%%%%%%%%%%%%%%%%%%%%%%%%%%%%%%%%%%%%%%%%%%%%%%%%%%%%%
\frame{
  \frametitle{AMEN varying $K$ Net Dependence}

  \includegraphics[width=1\textwidth]{netPerfCoef_ameSR}
}
%%%%%%%%%%%%%%%%%%%%%%%%%%%%%%%%%%%%%%%%%%%%%%%%%%%%%%%%%%%%

\end{document}
