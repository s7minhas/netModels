\documentclass[12pt]{letter}
\usepackage{graphicx,fullpage}
\name{Dr. Michael D. Ward}
\signature{}%\includegraphics{/Users/mw160/Documents/ImagesSigs/mdwsig
\address{} %\includegraphics[scale=.3]{}%drawing.pdf}
\usepackage[T1]{fontenc}
\usepackage[default,osfigures,scale=0.95]{opensans}
\usepackage{ae}
\begin{document}
\begin{letter}
{Professor William Jacoby, Editor\\
American Journal of Political Science\\
Department of Political Science\\
Michigan State University\\
sent electronically with submission}


\opening{Dear Bill:}

We have submitted to AJPS a manuscript entitled ``Amen for Latent Factor Models,'' co-authored by Shahryar Minhas, Peter Hoff, and Michael D. Ward.  We think this will make an important contribution to the study of political networks.  It was motivated by two basic goals: 1) to illustrate the benefits of using a latent factor approach to study hard to study networks, with missing and unobserved data, the structure of which is not known beforehand and 2) to correct some of the misinformation found in recent writings on the ERGM model, including the article by Skyler Cranmer and Bruce Desmarais which is appearing online now (2016).

The Cranmer and Desmarais' forthcoming article  (and others they have published with the same points)  have focused on showing the superiority of ERGM approaches over other approaches, including the Latent Space Model.  But the latent space models they excoriate are not in widespread use. Instead, the bulk of political science uses latent factor models. These latter models do not have the same weaknesses that Cranmer and Desmarais are pointing out. Indeed, they explicitly avoid them.  This is overlooked and misrepresented in the AJPS article that is forthcoming. Moreover, theses models have developed quite a bit beyond the genre that is criticized therein.  

We show that the developments over the last decade in this approach--which are ignored by Cranmer and Desmarais--present a principled way of doing inference on network models. This inference allows additive and multiplicative terms, permits multiplex networks, embraces longitudinal, dynamic networks, all within a regression framework that permits binary, ordinal and continuous data, and is familiar to political scientists: the general linear model. Finally, these newer models (AMEN models) not only have a simple software platform to implement them, but they are much more accurate in in-and-out of sample forecasts than other approaches, including ERGM.  All approaches do reasonably well at predicting the zeros in sparse networks, but the latent factor models are vastly superior at predicting the actual linkages. 

Our hope is that you and your reviewers will agree with our assessment. While we recognize the constraints you face, we further hope that after diligent review, this manuscript may appear online and in print in a timely fashion to help expand the use of network models, and to correct the mistaken inferences about latent space models that recent writing has created among readers of the AJPS. That said, we have constructed this as a broader contribution of interest to the readers of AJPS, rather than a more limited critique to the aforementioned article.

\closing{Respectfully submitted on behalf of Shahryar Minhas and Peter Hoff,} \vspace{.1in}

%\encl{}

\end{letter}
\end{document}\bye
