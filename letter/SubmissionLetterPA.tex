\documentclass[10pt]{letter}
\usepackage{graphicx,fullpage}
\name{}
\signature{}%\includegraphics{/Users/mw160/Documents/ImagesSigs/mdwsig
\address{} %\includegraphics[scale=.3]{}%drawing.pdf}
\usepackage[T1]{fontenc}
\usepackage[default,osfigures,scale=0.95]{opensans}
\usepackage{ae}
\begin{document}
\begin{letter}
{Professor Jeff Gill, Editor\\
Political Analysis\\
Department of Political Science\\
American University\\
sent electronically with submission}


\opening{Dear Professor Gill \& colleagues:}

 % and the general bilinear mixed effects model introduced in \textit{Political Analysis} (Hoff \& Ward, 2004)
We herewith submit to \textit{Political Analysis} a manuscript entitled ``Inferential Approaches for Network Analysis: AMEN for Latent Factor Models,'' co-authored by Shahryar Minhas, Peter Hoff, and Michael D. Ward.  We think this will make an important contribution to the study of political networks.  The Additive and Multiplicative Effects (AME) model further develops the latent space model originally introduced in the \textit{Journal of American Statistical Association} (Hoff et al. 2002). The new developments allow us to more accurately estimate higher-order dependencies within a Bayesian regression based framework that is familiar to most scholars. We compare AME with the latent space model and the exponential random graph model (ERGM).   

We show that the AME model presents a principled way of doing inference on networks. It is general and can be used on binary, ordinal, and continuous data from longitudinal networks or snapshots. It is built on a framework familiar to many political scientists: the generalized linear model. Finally, we focus on an application to show that this model is much more accurate in making out-of-sample forecasts than extant approaches.  All three approaches do reasonably well at predicting the zeros in sparse networks, but the AME model is vastly superior at predicting occurrences of actual linkages. In sum, we argue that the AME framework offers a principled way to measure the underlying data generating process characterizing social networks. 

Among the associate editors, Xun Pang is knowledgable about this approach having seen it presented at the 2017 Asian Political Methodology meetings held in Seoul in January.  

\closing{Respectfully submitted,\\~\\ Shahryar Minhas\\Peter Hoff\\Michael D. Ward} \vspace{.1in}

%\encl{}

\end{letter}
\end{document}\bye
